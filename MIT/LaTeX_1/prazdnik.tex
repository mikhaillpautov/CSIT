На крыше одного дома сидели два чертежника и ели гречневую кашу.

Вдруг один из чертежников радостно вскрикнул и достал из кармана длинный носовой платок. Ему пришла в голову блестящая идея"--- завязать в кончик платка двадцатикопеечную монетку и швырнуть все это с крыши вниз на улицу, и посмотреть, что из этого получится.

Второй чертежник, быстро уловив идею первого, доел гречневую кашу, высморкался и, облизав себе пальцы, принялся наблюдать за первым чертежником.

Однако внимание обоих чертежников было отвлечено от опыта с платком и двадцатикопеечной монеткой. На крыше, где сидели оба чертежника, произошло событие, не могущее быть незамеченным.

Дворник Ибрагим приколачивал к трубе длинную палку с выцветшим флагом.

Чертежники спросили Ибрагима, что это значит, на что Ибрагим отвечал: <<Это значит, что в городе праздник>>. 

"--* А какой же праздник, Ибрагим?"--~ спросили чертежники.

"--* А праздник такой, что наш любимый поэт сочинил новую поэму,"--~ сказал Ибрагим.

И чертежники, устыженные своим незнанием, растворились в воздухе.

\begin{flushright}
9 января 1935
\end{flushright}