Иван Яковлевич Бобов проснулся в самом приятном настроении духа. Он выглянул из"=под одеяла и сразу увидел потолок. Потолок был украшен большим серым пятном с зеленоватыми краями. Если смотреть на пятно пристально, одним глазом, то пятно становится похоже на носорога, запряженного в тачку, хотя другие находили, что оно больше походит на трамвай, на котором верхом сидит великан,"---  а впрочем, в этом пятне можно было усмотреть очертания даже какого"=то города. Иван Яковлевич посмотрел на потолок, но не в то место, где было пятно, а так, неизвестно куда; при этом он улыбнулся и сощурил глаза. Потом он вытаращил глаза и так высоко поднял брови, что лоб сложился, как гармошка, и чуть совсем не исчез, если бы Иван Яковлевич не сощурил глаза опять, и вдруг, будто устыдившись чего"=то, натянул одеяло себе на голову. Он сделал это так быстро, что из"=под другого конца одеяла выставились голые ноги Ивана Яковлевича, и сейчас же на большой палец левой ноги села муха. Иван Яковлевич подвигал этим пальцем, и муха перелетела и села на пятку. Тогда Иван Яковлевич схватил одеяло обеими ногами, одной ногой он подцепил одеяло снизу, а другую ногу вывернул и прижал ею одеяло сверху, и таким образом стянул одеяло со своей головы."--*  Шиш,"--~ сказал Иван Яковлевич и надул щеки. Обыкновенно, когда Ивану Яковлевичу что"=нибудь удавалось или, наоборот, совсем не выходило, Иван Яковлевич всегда говорил <<шиш>>"--- разумеется, не громко и вовсе не для того, чтобы кто"=нибудь это слышал, а так, про себя, самому себе. И вот, сказав <<шиш>>, Иван Яковлевич сел на кровать и протянул руку к стулу, на котором лежали его брюки, рубашка и прочее белье. Брюки Иван Яковлевич любил носить полосатые. Но раз, действительно, нигде нельзя было достать полосатых брюк. Иван Яковлевич и в <<Ленинградодежде>> был, и в Универмаге, и в Гостином дворе, и на Петроградской стороне обошел все магазины. Даже куда"=то на Охту съездил, но нигде полосатых брюк не нашел. А старые брюки Ивана Яковлевича износились уже настолько, что одеть их стало невозможно. Иван Яковлевич зашивал их несколько раз, но наконец и это перестало помогать. Иван Яковлевич обошел все магазины и, опять не найдя нигде полосатых брюк, решил наконец купить клетчатые. Но и клетчатых брюк нигде не оказалось. Тогда Иван Яковлевич решил купить себе серые брюки, но и серых нигде себе не нашел. Не нашлись нигде и черные брюки, годные на рост Ивана Яковлевича. Тогда Иван Яковлевич пошел покупать синие брюки, но, пока он искал черные, пропали всюду и синие и коричневые. И вот, наконец, Ивану Яковлевичу пришлось купить зеленые брюки с желтыми крапинками. В магазине Ивану Яковлевичу показалось, что брюки не очень уж яркого цвета и желтая крапинка вовсе не режет глаз. Но, придя домой, Иван Яковлевич обнаружил, что одна штанина и точно будто благородного оттенка, но зато другая просто бирюзовая, и желтая крапинка так и горит на ней. Иван Яковлевич попробовал вывернуть брюки на другую сторону, но там обе половины имели тяготение перейти в желтый цвет с зелеными горошинами и имели такой веселый вид, что, кажись, вынеси такие штаны на эстраду после сеанса кинематографа, и ничего больше не надо: публика полчаса будет смеяться. Два дня Иван Яковлевич не решался надеть новые брюки, но когда старые разодрались так, что издали можно было видеть, что и кальсоны Ивана Яковлевича требуют починки, пришлось надеть новые брюки. Первый раз в новых брюках Иван Яковлевич вышел очень осторожно. Выйдя из подъезда, он посмотрел раньше в обе стороны и, убедившись, что никого поблизости нет, вышел на улицу и быстро зашагал по направлению к своей службе. Первым повстречался яблочный торговец с большой корзиной на голове. Он ничего не сказал, увидя Ивана Яковлевича, и только, когда Иван Яковлевич прошел мимо, остановился, и так как корзина не позволила повернуть голову, то яблочный торговец повернулся весь сам и посмотрел вслед Ивану Яковлевичу,"--- может быть, покачал бы головой, если бы опять"=таки не все та же корзина. Иван Яковлевич бодро шел вперед, считая свою встречу с торговцем хорошим предзнаменованием. Он не видел маневра торговца и утешал себя, что брюки не так уж бросаются в глаза. Теперь навстречу Ивану Яковлевичу шел такой же служащий, как и он, с портфелем под мышкой. Служащий шел быстро, зря по сторонам не смотрел, а больше себе под ноги. Поравнявшись с Иваном Яковлевичем, служащий скользнул взглядом по брюкам Ивана Яковлевича и остановился. Иван Яковлевич остановился тоже. Служащий смотрел на Ивана Яковлевича, а Иван Яковлевич на служащего.

"--* Простите,"--~ сказал служащий,"--* вы не можете сказать мне, как пройти в сторону этого\dots государственного\dots биржи? 

"--* Это вам надо идти по мостовой\dots по мосту\dots нет, надо идти так, а потом так,"--~ сказал Иван Яковлевич.

Служащий сказал спасибо и быстро ушел, а Иван Яковлевич сделал несколько шагов вперед, но, увидев, что теперь навстречу ему идет не служащий, а служащая, опустил голову и перебежал на другую сторону улицы. На службу Иван Яковлевич пришел с опозданием и очень злой. Сослуживцы Ивана Яковлевича, конечно, обратили внимание на зеленые брюки со штанинами разного оттенка, но, видно, догадались, что это"---  причина злости Ивана Яковлевича, и расспросами его не беспокоили. Две недели мучился Иван Яковлевич, ходя в зеленых брюках, пока один из его сослуживцев, Аполлон Максимович Шилов не предложил Ивану Яковлевичу купить полосатые брюки самого Аполлона Максимовича, будто бы не нужные Аполлону Максимовичу.

\begin{flushright}
1934--1937
\end{flushright}